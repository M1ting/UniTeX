
\definition-\lemma. Let F be a subfield of a field L. An element a \in L is called 
algebraic over F if one of the following equivalent conditions is satisfied 
  (1) f(a) = 0 for a non-zero polynomial f(X) \in F[X]
  (2) elements 1, a, a^2, \cdots are linearly dependent over F
  (3) F-vector space F[a] = \{ \sum a_ia^i : a_i \in F \} is of finite dimension over F 
  (4) F[a] = F(a)

\proof. Omitted.

  For an element a algebraic over F denote by
                                f_a(X) \in F[X]
the monic polynomial of minimal degree such that f_a(a) = 0. 
  This polynomial is irreducible. f_a(X) is a linear polynomial iff a \in F.

\lemma. Define a ring homomorphism F[X] \rarr L, g(X) \rarr g(a). Its kernel is the
  principal ideal generated by f_a(X) and its image is F(a), so
                              F[X]/(f_a(X)) \cong F(a)
\proof. Using the division algorithm.
                            
\definition. A field L is called algebraic over its subfield F if every element of L is
algebraic over F. The extension L/F is called algebraic. 

\definition. Let F be a subfield of a field L. The dimension of L as a vector space
over F is called the degree [L : F] of the extension L/F.

  If a is algebraic over F then [F(a) : F] is finite and it equals the degree of the
monic irreducible polynomial fa of a over F.


