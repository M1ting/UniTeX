
  \S 1. The periodicity of classical modular function j can be 
described algebraically by saying that j(\tau) = j(\frac{a\tau+b}{c\tau+d}) 
for any a, b, c, d \in \Z with ad - bc = 1.

\theorem (Kronecker). for any integer \tau in the quadratic field \Q(\sqrt{-D}), 
j(\tau) is an algebraic integer whose degree is the class number of \Q(\sqrt{-D})

\example. for \Q(\sqrt{-1}), class number = 1, j(-1) = 12^3.
\example. for \Q(\sqrt{-163}), class number = 1, j((1+\sqrt{-163})/2) = -640320^3.

% \frac{-b\pm\sqrt{b^2-4ac}}{2a}
